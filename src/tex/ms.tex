% Define document class
\documentclass[twocolumn]{aastex631}
\usepackage{showyourwork}

% Begin!
\begin{document}

% Title
\title{Case study: The corona of an M7 fast rotator with a high latitude flare}

% Author list
\author{Ekaterina Ilin, Katja Poppenh\"ager}


% Abstract with filler text
\begin{abstract}
    Small scale fields

    late M dwarfs

    XMM-Newton observations

    Flare

    Anything unusual about the dwarf
    
\end{abstract}

% Main body with filler text
\section{Introduction}
\label{sec:intro}

Small scale fields of late M dwarfs

We know little about the corona of late M dwarfs

Planets around late M dwarfs exists, see TRAPPIST-1 or LHS 3154~\citep{stefansson2023extreme}

TIC 277 -- edge-on rotation, high-latitude flare

Age unknown ... \citep{schneider2018hazmat} classify it as field star, but its FUV and NUV fluxes are on the high end for field stars and would be compatible with the NUV/FUV emission from younger open cluster members as well


\begin{table}

    \caption{TIC 277539431 (WISEA J105515.71-735611.3)}
    \begin{tabular}{ll}\hline 
         Parameter & Value  \\\hline
         Distance & $13.70\pm0.11\,$pc [2] \\
         $G$ mag & \\
         $G_{BP}$ mag & \\
         $G_{BP}$ mag & \\
         $K_s$ mag & $9.666 \pm 0.024$ [3]\\
         $J$ mag & $10.630 \pm 0.023$ [3] \\
         SpT & M7 [1]\\
         age & \\
         Rotation period & $273.618 \pm 0.007\,$min [1]\\
         $v\sin i$ & $38.6\pm1.0\,$km/s [1] \\
         Inclination & $87.0^{+2.0}_{-2.4}\,$ deg [1]\\
         Radius & $0.145\pm0.004\,R_\odot$ [1]\\
         TESS flaring rate $\beta$ &  \\\hline
        
    \end{tabular}
    \newline\footnotesize
    [1] \citet{ilin2021giant}, [2] \citet{bailer-jones2018estimating}, [3] 2MASS, \citet{skrutskie2006two}
    \label{tab:modelparams}
\end{table}


\section{Observations}

XMM-Newton

uninterrupted two rotation periods

MOS1/2 and PN, as well as OM data

\section{Results}

Figure~\ref{fig:lightcurves} shows the time series in X-ray and optical. 

\begin{figure*}[ht!]
    \script{FIGURE_lightcurves.py}
    \begin{centering}
        \includegraphics[width=0.75\linewidth]{figures/lightcurves.png}
        \caption{
         Top panel: Optical Monitoring (OM) light curve. Bottom panel: X-ray (PN, MOS1 and MOS2 combined) light curve. 
        }
        \label{fig:lightcurves}
    \end{centering}
\end{figure*}


show spectra with fits and residuals

\subsection{Rotational variability}

\begin{figure*}[ht!]
    \script{FIGURE_periodograms_xray_and_optical.py}
    \begin{centering}
        \includegraphics[width=0.75\linewidth]{figures/periodogram_xmm_stacked.png}
        \caption{
         MOS and PN light curves stacked Lomb-Scargle periodogram.
        }
        \label{fig:coherence_hist}
    \end{centering}
\end{figure*}

\begin{figure*}[ht!]
    \script{FIGURE_periodograms_xray_and_optical.py}
    \begin{centering}
        \includegraphics[width=0.75\linewidth]{figures/periodogram_om.png}
        \caption{
         OM light curve Lomb-Scargle periodogram.
        }
        \label{fig:coherence_hist}
    \end{centering}
\end{figure*}

periodograms

present simple model and upper limits from the data

\subsection{Coronal temperature and luminosity}

From the proposal: $F_X=2.7^{-13}$ erg/s/cm$^2$, $L_X=6\cdot10^27$ erg/s (update with new eROSITA data) -- 1 oom above the measured XMM values.
\begin{table*}
% \movetableright=-20mm
\footnotesize
    \script{TABLE_specfits.py}
    \caption{XSPEC fits to PN, MOS1/MOS2 and joint data also.}
    \input{output/specfit.tex}
        \label{tab:specfit}
\end{table*}

\begin{figure}
    \script{FIGURE_data_resid.py}
    \begin{centering}
        \includegraphics[width=\linewidth]{figures/pn_onlyflare_data_resid.png}
        \caption{
         Flare only PN Double apec spectrum fit.
        }
        \label{fig:spec_pn_onlyflare}
    \end{centering}
\end{figure}

\begin{figure}
    \script{FIGURE_data_resid.py}
    \begin{centering}
        \includegraphics[width=\linewidth]{figures/pn_noflare_data_resid.png}
        \caption{
         No flare PN Double apec spectrum fit.
        }
        \label{fig:spec_pn_noflare}
    \end{centering}
\end{figure}

\begin{figure}
    \script{FIGURE_data_resid.py}
    \begin{centering}
        \includegraphics[width=\linewidth]{figures/pn_data_resid.png}
        \caption{
         PN Double apec spectrum fit.
        }
        \label{fig:spec_pn}
    \end{centering}
\end{figure}

\begin{figure}
    \script{FIGURE_data_resid.py}
    \begin{centering}
        \includegraphics[width=\linewidth]{figures/mos1_data_resid.png}
        \caption{
         MOS1 Double apec spectrum fit.
        }
        \label{fig:spec_mos1}
    \end{centering}
\end{figure}

\begin{figure}
    \script{FIGURE_data_resid.py}
    \begin{centering}
        \includegraphics[width=\linewidth]{figures/mos2_data_resid.png}
        \caption{
         MOS2 Double apec spectrum fit.
        }
        \label{fig:spec_mos2}
    \end{centering}
\end{figure}

\begin{figure}
    \script{FIGURE_data_resid.py}
    \begin{centering}
        \includegraphics[width=\linewidth]{figures/joint_onlyflare_data_resid.png}
        \caption{
         Flare only joint Double APEC spectrum fit.
        }
        \label{fig:spec_joint_onlyflare}
    \end{centering}
\end{figure}


\begin{figure}
    \script{FIGURE_data_resid.py}
    \begin{centering}
        \includegraphics[width=\linewidth]{figures/joint_noflare_data_resid.png}
        \caption{
         Joint Double APEC spectrum fit with flare excluded.
        }
        \label{fig:spec_joint_noflare}
    \end{centering}
\end{figure}

\begin{figure}
    \script{FIGURE_data_resid.py}
    \begin{centering}
        \includegraphics[width=\linewidth]{figures/joint_all_data_resid.png}
        \caption{
         Joint Double APEC spectrum fit.
        }
        \label{fig:spec_joint_all}
    \end{centering}
\end{figure}


spectral analysis

APEC+APEC model, grsa abundances

\subsection{Flare}

optical and X-ray combined

\section{Discussion}
\citep{pass2022constraints} on rotational evol of late Ms

FFD power law fit converged after 16000 steps on $\alpha = 1.77_{-0.23}^{+0.29}$.

$\log_10(R_{31.5}$, that is, the rate of flares per day with energies above $\log E_{flare}=10^{31.5}\,$erg, is ca. $-0.7$, consistent with other late M dwarfs~\citep{medina2022galactic}.


\begin{figure}
    \script{FIGURE_tess_ffd.py}
    \begin{centering}
        \includegraphics[width=\linewidth]{figures/tess_ffd.png}
        \caption{
         Flare frequency distribution and power law fit.
        }
        \label{fig:ffd}
    \end{centering}
\end{figure}

\begin{figure}[ht!]
    \script{FIGURE_lx_lbol.py}
    \begin{centering}
        \includegraphics[width=\linewidth]{figures/lx_lbol.png}
        \caption{
         Rossby number versus $L_x/L_{bol}$.
        }
        \label{fig:lxlbol}
    \end{centering}
\end{figure}


\section{Summary and Conclusions}


\section{Acknowledgements}
\citep{lightkurvecollaboration2018lightkurve}

\section{Data Availability}


\bibliography{references}

\end{document}
