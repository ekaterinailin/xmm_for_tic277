% Define document class
\documentclass[twocolumn]{aastex631}
\usepackage{showyourwork}

% Begin!
\begin{document}

% Title
\title{Corona of an M7 fast rotator with a high latitude flare}

% Author list
\author{@ekaterinailin}

% Abstract with filler text
\begin{abstract}
    Small scale fields

    late M dwarfs

    XMM-Newton observations

    Flare

    Anything unusual about the dwarf
    
\end{abstract}

% Main body with filler text
\section{Introduction}
\label{sec:intro}

\section{Observations}

XMM-Newton

uninterrupted two rotation periods

MOS1/2 and PN, as well as OM data

\section{Results}

show light curve

show spectra with fits and residuals

\subsection{Rotational variability}

\begin{figure*}[ht!]
    \script{FIGURE_periodograms_xray_and_optical.py}
    \begin{centering}
        \includegraphics[width=\linewidth]{figures/periodogram_xmm_stacked.png}
        \caption{
         MOS and PN light curves stacked Lomb-Scargle periodogram.
        }
        \label{fig:coherence_hist}
    \end{centering}
\end{figure*}

\begin{figure*}[ht!]
    \script{FIGURE_periodograms_xray_and_optical.py}
    \begin{centering}
        \includegraphics[width=\linewidth]{figures/periodogram_om.png}
        \caption{
         MOS and PN light curves stacked Lomb-Scargle periodogram.
        }
        \label{fig:coherence_hist}
    \end{centering}
\end{figure*}

periodograms

present simple model and upper limits from the data

\subsection{Coronal temperature and luminosity}

spectral analysis

APEC+APEC model, grsa abundances

\subsection{Flare}

optical and X-ray combined

\section{Discussion}

Ro-Lx relation

\section{Summary and Conclusions}


\section{Acknowledgements}
\citep{lightkurvecollaboration2018lightkurve}

\section{Data Availability}


\bibliography{references}

\end{document}
