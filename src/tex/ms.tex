% Define document class
\documentclass[twocolumn]{aastex631}
\usepackage{showyourwork}

% Begin!
\begin{document}

% Title
\title{Corona of an M7 fast rotator with a high latitude flare}

% Author list
\author{Ekaterina Ilin, Katja Poppenh\"ager}


% Abstract with filler text
\begin{abstract}
    Small scale fields

    late M dwarfs

    XMM-Newton observations

    Flare

    Anything unusual about the dwarf
    
\end{abstract}

% Main body with filler text
\section{Introduction}
\label{sec:intro}

Small scale fields of late M dwarfs

We know little about the corona of late M dwarfs

Planets around late M dwarfs exists, see TRAPPIST-1 or LHS 3154~\citep{stefansson2023extreme}

TIC 277 -- edge-on rotation, high-latitude flare



\begin{table}

    \caption{TIC 277539431 (WISEA J105515.71-735611.3)}
    \begin{tabular}{ll}\hline 
         Parameter & Value  \\\hline
         Distance & $13.70\pm0.11\,$pc [2] \\
         $K_s$ mag & $9.666 \pm 0.024$ [3]\\
         $J$ mag & $10.630 \pm 0.023$ [3] \\
         SpT & M7 [1]\\
         Rotation period & $273.618 \pm 0.007\,$min [1]\\
         $v\sin i$ & $38.6\pm1.0\,$km/s [1] \\
         Inclination & $87.0^{+2.0}_{-2.4}\,$ deg [1]\\
         Radius & $0.145\pm0.004\,R_\odot$ [1]\\
         TESS flaring rate $\beta$ &  \\\hline
        
    \end{tabular}
    \newline\footnotesize
    [1] \citet{ilin2021giant}, [2] \citet{bailer-jones2018estimating}, [3] 2MASS, \citet{skrutskie2006two}
    \label{tab:modelparams}
\end{table}


\section{Observations}

XMM-Newton

uninterrupted two rotation periods

MOS1/2 and PN, as well as OM data

\section{Results}

Figure~\ref{fig:lightcurves} shows the time series in X-ray and optical. 

\begin{figure*}[ht!]
    \script{FIGURE_lightcurves.py}
    \begin{centering}
        \includegraphics[width=0.75\linewidth]{figures/lightcurves.png}
        \caption{
         Top panel: Optical Monitoring (OM) light curve. Bottom panel: X-ray (PN, MOS1 and MOS2 combined) light curve. 
        }
        \label{fig:lightcurves}
    \end{centering}
\end{figure*}


show spectra with fits and residuals

\subsection{Rotational variability}

\begin{figure*}[ht!]
    \script{FIGURE_periodograms_xray_and_optical.py}
    \begin{centering}
        \includegraphics[width=0.75\linewidth]{figures/periodogram_xmm_stacked.png}
        \caption{
         MOS and PN light curves stacked Lomb-Scargle periodogram.
        }
        \label{fig:coherence_hist}
    \end{centering}
\end{figure*}

\begin{figure*}[ht!]
    \script{FIGURE_periodograms_xray_and_optical.py}
    \begin{centering}
        \includegraphics[width=0.75\linewidth]{figures/periodogram_om.png}
        \caption{
         OM light curve Lomb-Scargle periodogram.
        }
        \label{fig:coherence_hist}
    \end{centering}
\end{figure*}

periodograms

present simple model and upper limits from the data

\subsection{Coronal temperature and luminosity}


\begin{figure}[ht!]
    \script{FIGURE_data_resid.py}
    \begin{centering}
        \includegraphics[width=\linewidth]{figures/pn_onlyflare1_data_resid.png}
        \caption{
         Flare only PN Double apec spectrum fit.
        }
        \label{fig:spec_pn_onlyflare}
    \end{centering}
\end{figure}


spectral analysis

APEC+APEC model, grsa abundances

\subsection{Flare}

optical and X-ray combined

\section{Discussion}

Ro-Lx relation

\section{Summary and Conclusions}


\section{Acknowledgements}
\citep{lightkurvecollaboration2018lightkurve}

\section{Data Availability}


\bibliography{references}

\end{document}
