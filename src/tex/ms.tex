% Define document class
\documentclass[twocolumn]{aastex631}
\usepackage{showyourwork}

% Begin!
\begin{document}

% Title
\title{Does a fully convective star need a special corona to produce a polar flare?}

% Author list
\author{Ekaterina Ilin, Katja Poppenh\"ager, Beate Stelzer}


% Abstract with filler text
\begin{abstract}
    TIC 277 is currently the star with the highest latitude flare located to date. In 2020, TESS observed this rapidly rotating M7 dwarf produce a flare at 81 deg latitude. This is in stark contrast to solar flares that occur much closer to the equator, typically below 30 deg. Does a star require a special corona to produce polar flares?

    We obtained 36 ks ob XMM-Newton observations, and determined that the X-ray luminosity $L_X$ is slightly lower but not far below the $L_X$ of other stars with similar spectral types a rotation rates. We constructed flare frequency distributions using the three TESS Sectors, with a power law slope of $XXXX\pm XXXX$, and flare rate above $\log_{10}E=31.5\,$erg of XXXX, again typical of similar stars. Our results suggest that the star's magnetic activity is typical for its rotation rate and mass. However, the rotation period of 4.56 hours is on the short end, which could indicate that polar flares on these stars are a feature of youth.  
    
\end{abstract}

% Main body with filler text
\section{Introduction}
\label{sec:intro}
The smallest stars are crawling with rocky planets. The disk mass around the smallest stars decreases such that giant planets are no longer efficiently formed, leaving room for smaller objects in the orbit~(XXXX). Terrestrial planets in the habitable zones of these stars are a factor of XXXX more likely than around higher mass stars~\citep{hardegree-ullman2019kepler}. However, to be habitable in an Earth-like sense, merely the possibility of surface water in its liquid form does not suffice. Space weather of the host, that is, the high energy radiation and particles that force the planet's atmosphere, is another important factor. If the energetic particle flux is too high, the atmosphere may blow off. If not energetic enough, life may not emerge in the first place.

At the lowest masses, a star's evolution unfolds much slower than for a Sun. While the Sun's magnetic activity, and with it the flares, winds, coronal mass ejections and energetic particle eruption that make up space weather, decline rapidly over a few hundred megayears, a fully convective M dwarf stays highly active for gigayears. Add that the zone where liquid water can reside is much closer to these faint stars, and the result is a planet that may be exposed to violent space weather conditions for a good fraction of the age of the universe. 

Space weather originates from the stellar corona. The transition from star to brown dwarf is characterized by increasing rotation speed and decreasing coronal emission. The simultaneous surge of radio emission indicates a transition from stellar corona to planet-like magnetosphere at the bottom of the main sequence. Yet, brown dwarfs are regularly detected with energetic flares regardless of their low X-ray luminosity. How these flares are produced in the apparent absence of a solar-type corona is unclear. An intermittent corona is conceivable, similarly XYZ. Perhaps, the rapid rotation of these low-mass objects plays a role, as it governs the dynamo process within, and consequently influences the complex process of flux emergence~\citep{weber2023understanding}.   

\citet{ilin2021giant} searched fully convective dwarfs observed in TESS, and found four flares on four rapidly rotating ($P<10\,$h) M5-M7 dwarfs. These flares' durations spanned multiple rotation periods of each star, so that the flares were observed to rotate in and out view. The shape of this modulation together with known inclination (combining $P$ and $v \sin i$ from high resolution spectroscopy) allowed them to determine the latitudes of these flares. Surprisingly, these flares were found closer to the pole than to the equator, in contrast to the Sun, where flares are usually found below 30 deg latitude. Their results indicate a preference for flares at high latitudes in these star, as a chance finding was unlikely ($0.1\%$). If flares and associated particle eruptions have a latitude preference, the space weather of a planet in an aligned orbit would be less severe than previously thought. %Are high flaring latitudes  an effect of the dynamo at high rotation speeds? Or is there something special about these stars' coronae that allowed them to produce flares at these latitudes? 

%Planets around late M dwarfs exists, see TRAPPIST-1 or LHS 3154~\citep{stefansson2023extreme}

TIC 277539431, TIC 277 for short, is one out of four stars with a high-latitude flare in its TESS observations~\citep{ilin2021giant}. In fact, it showed the highest latitude flare known to date at $81\pm1\,$deg. TIC 277 is an M7 dwarf with a rotation period of 4.56 hours. Its age is unknown. \citet{schneider2018hazmat} classify it as a field star, but its FUV and NUV fluxes are on the high end for field stars and would be compatible with the NUV/FUV emission from younger open cluster members, as well. 
Is its high flaring latitude an effect of the dynamo at high rotation speeds? Or is there something special about this star's corona that led it to produce are flare at these latitudes? 

In this work, we follow TIC 277 with XMM-Newton for 36 ks, derive its coronal properties, and investigate if producing a high-latitude flare requires an unusual corona, or if it is rather typical of stars of this spectral type and rotation.


\begin{table}

    \caption{TIC 277539431 (WISEA J105515.71-735611.3)}
    \begin{tabular}{ll}\hline 
         Parameter & Value  \\\hline
         Distance & $13.70\pm0.11\,$pc [2] \\
         $G$ mag & \\
         $G_{BP}$ mag & \\
         $G_{BP}$ mag & \\
         $K_s$ mag & $9.666 \pm 0.024$ [3]\\
         $J$ mag & $10.630 \pm 0.023$ [3] \\
         SpT & M7 [1]\\
         age & \\
         Rotation period & $273.618 \pm 0.007\,$min [1]\\
         $v\sin i$ & $38.6\pm1.0\,$km/s [1] \\
         Inclination & $87.0^{+2.0}_{-2.4}\,$ deg [1]\\
         Radius & $0.145\pm0.004\,R_\odot$ [1]\\\hline
        
    \end{tabular}
    \newline\footnotesize
    [1] \citet{ilin2021giant}, [2] \citet{bailer-jones2018estimating}, [3] 2MASS, \citet{skrutskie2006two}
    \label{tab:modelparams}
\end{table}


\section{Observations}

\subsection{XMM-Newton}
XMM-Newton

uninterrupted two rotation periods

MOS1/2 and PN, as well as OM data

\subsection{TESS}

\section{Methods}
\label{sec:methods}

\subsection{Flux and luminosity}

\subsection{Spectral fitting}

\subsection{Flares}

\subsubsection{XMM-Newton flare}

X-ray energy of $8.4\pm0.7\cdot10^{30}\,$erg

\subsubsection{TESS flares}

\section{Results}

Figure~\ref{fig:lightcurves} shows the time series in X-ray and optical. 

\begin{figure*}[ht!]
    \script{FIGURE_lightcurves.py}
    \begin{centering}
        \includegraphics[width=0.75\linewidth]{figures/lightcurves.png}
        \caption{
         Top panel: Optical Monitoring (OM) light curve. Bottom panel: X-ray (PN, MOS1 and MOS2 combined) light curve. 
        }
        \label{fig:lightcurves}
    \end{centering}
\end{figure*}


show spectra with fits and residuals

% \subsection{Rotational variability}

% \begin{figure*}[ht!]
%     \script{FIGURE_periodograms_xray_and_optical.py}
%     \begin{centering}
%         \includegraphics[width=0.75\linewidth]{figures/periodogram_xmm_stacked.png}
%         \caption{
%          MOS and PN light curves stacked Lomb-Scargle periodogram.
%         }
%         \label{fig:coherence_hist}
%     \end{centering}
% \end{figure*}

% \begin{figure*}[ht!]
%     \script{FIGURE_periodograms_xray_and_optical.py}
%     \begin{centering}
%         \includegraphics[width=0.75\linewidth]{figures/periodogram_om.png}
%         \caption{
%          OM light curve Lomb-Scargle periodogram.
%         }
%         \label{fig:coherence_hist}
%     \end{centering}
% \end{figure*}

% periodograms

% present simple model and upper limits from the data

% how large of a loop would that be? convert relative amplitude to flux, convert flux to emission measure, convert emission measure to loop size?

\subsection{Coronal temperature and luminosity}

From the proposal: $F_X=2.7^{-13}$ erg/s/cm$^2$, $L_X=6\cdot10^27$ erg/s (update with new eROSITA data) -- 1 oom above the measured XMM values.
\begin{table*}
% \movetableright=-20mm
\footnotesize
    \script{TABLE_specfits.py}
    \caption{XSPEC fits to PN, MOS1/MOS2 and joint data also.}
    \input{output/specfit.tex}
        \label{tab:specfit}
\end{table*}

\begin{figure}
    \script{FIGURE_data_resid.py}
    \begin{centering}
        \includegraphics[width=\linewidth]{figures/pn_onlyflare_data_resid.png}
        \caption{
         Flare only PN Double apec spectrum fit.
        }
        \label{fig:spec_pn_onlyflare}
    \end{centering}
\end{figure}

\begin{figure}
    \script{FIGURE_data_resid.py}
    \begin{centering}
        \includegraphics[width=\linewidth]{figures/pn_noflare_data_resid.png}
        \caption{
         No flare PN Double apec spectrum fit.
        }
        \label{fig:spec_pn_noflare}
    \end{centering}
\end{figure}

\begin{figure}
    \script{FIGURE_data_resid.py}
    \begin{centering}
        \includegraphics[width=\linewidth]{figures/pn_data_resid.png}
        \caption{
         PN Double apec spectrum fit.
        }
        \label{fig:spec_pn}
    \end{centering}
\end{figure}

\begin{figure}
    \script{FIGURE_data_resid.py}
    \begin{centering}
        \includegraphics[width=\linewidth]{figures/mos1_data_resid.png}
        \caption{
         MOS1 Double apec spectrum fit.
        }
        \label{fig:spec_mos1}
    \end{centering}
\end{figure}

\begin{figure}
    \script{FIGURE_data_resid.py}
    \begin{centering}
        \includegraphics[width=\linewidth]{figures/mos2_data_resid.png}
        \caption{
         MOS2 Double apec spectrum fit.
        }
        \label{fig:spec_mos2}
    \end{centering}
\end{figure}

\begin{figure}
    \script{FIGURE_data_resid.py}
    \begin{centering}
        \includegraphics[width=\linewidth]{figures/joint_onlyflare_data_resid.png}
        \caption{
         Flare only joint Double APEC spectrum fit.
        }
        \label{fig:spec_joint_onlyflare}
    \end{centering}
\end{figure}


\begin{figure}
    \script{FIGURE_data_resid.py}
    \begin{centering}
        \includegraphics[width=\linewidth]{figures/joint_noflare_data_resid.png}
        \caption{
         Joint Double APEC spectrum fit with flare excluded.
        }
        \label{fig:spec_joint_noflare}
    \end{centering}
\end{figure}

\begin{figure}
    \script{FIGURE_data_resid.py}
    \begin{centering}
        \includegraphics[width=\linewidth]{figures/joint_all_data_resid.png}
        \caption{
         Joint Double APEC spectrum fit.
        }
        \label{fig:spec_joint_all}
    \end{centering}
\end{figure}


spectral analysis

APEC+APEC model, grsa abundances

\subsection{Flares}

optical and X-ray combined

FFD power law fit converged after 16000 steps on $\alpha = 1.77_{-0.23}^{+0.29}$.


\section{Discussion}
\label{sec:discussion}
% \citep{pass2022constraints} on rotational evol of late Ms

Is the magnetic activity of the star with the highest known flare latitude known to date is any way unusual? The star is rotating rapidly, even for the typically fast rotating late M dwarfs~\citep{medina2022galactic}. This could lead to an unusual manifestation of the stellar dynamo that produces high latitude flares that other dynamo states cannot generate. With TESS and XMM-Newton, we can now evaluate if this suspected exceptionality is reflected in the star's magnetic activity and coronal properties. We compare its flaring activity, coronal luminosity and temperatures to stars of similar spectral type and rotation rates

\subsection{Flaring activity}
\label{sec:discussion:flares}

$\log_10(R_{31.5}$, that is, the rate of flares per day with energies above $\log E_{flare}=10^{31.5}\,$erg, is ca. $-1.XXXX$, consistent with other saturated late M dwarfs~\citep{medina2022galactic}.

Compared to the X-ray luminosities of M7 dwarfs among the late M dwarfs observed by eROSITA in \citet{stelzer2022first}, TIC 277 appears mildly underluminous. This could be an effect of its fast rotation, and consequent supersaturation as in \citet{magaudda2022first}, but we lack the rotation periods to confirm this. 

\begin{figure}
    \script{FIGURE_tess_ffd.py}
    \begin{centering}
        \includegraphics[width=\linewidth]{figures/277539431_tess_ffd.png}
        \caption{
         Flare frequency distribution and power law fit.
        }
        \label{fig:ffd}
    \end{centering}
\end{figure}


\begin{figure}
    \script{FIGURE_r315_comparison.py}
    \begin{centering}
        \includegraphics[width=\linewidth]{figures/r315_prot.png}
        \caption{
         R31.5 against rotation period for fully convective M dwarfs.
        }
        \label{fig:r315}
    \end{centering}
\end{figure}


\subsection{X-ray luminosity}
\label{sec:discussion:xraylum}

\begin{figure}[ht!]
    \script{FIGURE_lx_lbol.py}
    \begin{centering}
        \includegraphics[width=\linewidth]{figures/lx_lbol.png}
        \caption{
         Rossby number versus $L_x/L_{bol}$.
        }
        \label{fig:lxlbol}
    \end{centering}
\end{figure}


The $L_X/L_{\rm bol}$ distributions in \citet{cook2014trends, wright2016solartype} grow sparse around $Ro\sim10^{-3}$. Using eROSITA data, \citet{magaudda2022first} cover both fully convective and rapidly rotating M dwarfs, and find a downward trend with decreasing $Ro$, pointing toward a supersaturation of the corona at fast rotation speeds, as has been conjectured before~\citep{jeffries2011investigating,ramsay2020tess}. \citet{jeffries2011investigating} suggest that the cause for supersaturation is centrifugal stripping of the corona, but \citet{reiners2022magnetism} show that if magnetic field increases, X-ray increases likewise with no sign of a break. They conclude that the trend is rather due to the underlying dynamo.

\cite{brown2023coronal}

\cite{johnstone2012soft}


\subsection{Coronal temperature}
\label{sec:discussion:xraytemp}




\section{Summary and Conclusions}


\section*{Acknowledgements}
\citep{lightkurvecollaboration2018lightkurve}
Based on observations obtained with XMM-Newton, an ESA science mission with instruments and contributions directly funded by ESA Member States and NASA.

DLR
\section*{Data Availability}


\bibliography{references}

\end{document}
