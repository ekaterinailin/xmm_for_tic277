% Define document class
\documentclass[twocolumn]{aastex631}
\usepackage{showyourwork}



% Begin!
\begin{document}


\newcommand{\Tcool}{\input{output/T1.tex}}
\newcommand{\Thot}{\input{output/T2.tex}}
\newcommand{\eom}{\input{output/om_flare.tex}}
\newcommand{\eepic}{\input{output/epic_flare.tex}}
\newcommand{\xraydur}{\input{output/epic_flare_dur.tex}}
\newcommand{\Tqmean}{\input{output/Tqmean.tex}}
\newcommand{\Tfmean}{\input{output/Tfmean.tex}}
\newcommand{\FX}{\input{output/epic_flux.tex}}
\newcommand{\LXquiet}{\input{output/epic_Lx_quiet.tex}}
\newcommand{\LXflaring}{\input{output/epic_Lx_flaring.tex}}
\newcommand{\Lbol}{\input{output/Lbol.tex}}
\newcommand{\LXLbol}{\input{output/lxlbol}}
\newcommand{\ffdalpha}{\input{output/tess_ffd_alpha}}
\newcommand{\ffdbeta}{\input{output/tess_ffd_beta}}
\newcommand{\ffdr}{\input{output/R315}}


% Title
\title{The corona of a fully convective star with a near-polar flare}

% Author list with  \affiliation

\author[0000-0002-6299-7542]{Ekaterina Ilin}
\affiliation{Leibniz Insitute for Astrophysics Potsdam (AIP), An der Sternwarte 16, 14482 Potsdam, Germany}

\author[0000-0003-1231-2194]{Katja Poppenh\"ager}
\affiliation{Leibniz Insitute for Astrophysics Potsdam (AIP), An der Sternwarte 16, 14482 Potsdam, Germany}
\affiliation{Institute for Physics and Astronomy, University of Potsdam, Karl-Liebknecht-Strasse 24/25, 14476 Potsdam, Germany}

\author{Beate Stelzer}
\affiliation{Institut für Astronomie \& Astrophysik, Eberhard-Karls-Universität Tübingen, Sand 1, 72076 Tübingen, Germany}
\affiliation{INAF – Osservatorio Astronomico di Palermo, Piazza del Parlamento 1, 90134 Palermo, Italy}



% Abstract with filler text
\begin{abstract}
    In 2020, the \textit{Transiting Exoplanet Survey Satellite} (\textit{TESS}) observed a rapidly rotating M7 dwarf, TIC~277539431, produce a flare at $81^{\circ}$ latitude, the highest latitude flare located to date. This is in stark contrast to solar flares that occur much closer to the equator, typically below $30^{\circ}$. The circumstances that allow flares at high latitudes to occur are poorly understood. We studied five Sectors of \textit{TESS} monitoring, and obtained 36 ks of \textit{XMM-Newton} observations, to find that TIC~277539431's corona does not differ significantly from other low mass stars on the \textcolor{red}{canonical saturated activity branch with respect to coronal temperatures and flaring activity, but shows lower luminosity in soft X-ray emission by about an order of magnitude, consistent with other late M dwarfs.} Combining the lack of X-ray flux, the high latitude flare, the star's viewing geometry, and the otherwise typical stellar corona, we suggest that the migration of flux emergence to the poles in rapid rotators can explain the observations of TIC~277539431.
    
    
    %Additionally, XMM-Newton captured a simultaneous X-ray and optical flare on TIC 277 with a bolometric energy of \eom and a soft X-ray energy of \eepic in the 0.2-2 keV band. In combination with the TESS data, we infer that TIC 277 produces flares at or above this energy approximately once a day. 
    %Our results indicate that TIC 277 is faint in X-ray copX-ray and flaring activity is representative of other rapidly rotating late M dwarfs, but that further measurements of localized flares are needed to understand the conditions under which high latitude flares in these stars can occur.
    
%    and determined that the X-ray luminosity $L_X$ is slightly lower but not far below the $L_X$ of other stars with similar spectral types a rotation rates. We constructed flare frequency distributions using the three TESS Sectors, with a power law slope of $XXXX\pm XXXX$, and flare rate above $\log_{10}E=31.5\,$erg of XXXX, again typical of similar stars. Our results suggest that the star's magnetic activity is typical for its rotation rate and mass. However, the rotation period of 4.56 hours is on the short end, which could indicate that polar flares on these stars are a feature of youth.  
    
\end{abstract}

% Main body with filler text
\section{Introduction}
\label{sec:intro}
Most M dwarfs hosts terrestrial planets, with a sizable fraction orbiting at instellations where liquid water can exist~\citep{dressing2015occurrence, hardegree-ullman2019kepler, ment2023occurrence}. However, to be habitable in an Earth-like manner, merely the possibility of surface water in its liquid form does not suffice. Space weather of the host, that is, the high energy radiation and particles that impact the planet's atmosphere, affect its hospitality for life, for M dwarfs in particular~\citep{airapetian2020impact}. If the energetic (photon or particle) flux is too high, the atmosphere may blow off~\citep[e.g., ][]{lammer2003atmospheric, garraffo2017threatening, ketzer2022influence} and water may be lost from the surface~\citep{doamaral2022contribution}. If not energetic enough, life may not emerge in the first place~\citep{rimmer2018origin}.

In M dwarfs, stellar activity evolution on the main sequence unfolds much more slowly than for \textcolor{red}{FGK stars}. While for the latter, magnetic activity, and with it the flares, winds, coronal mass ejections and energetic particle eruptions that make up their space weather, decline rapidly over a few hundred megayears, a fully convective M dwarf stays highly active for gigayears~\citep{magaudda2020relation, johnstone2021active, medina2022galactic}. Moreover, the zone where liquid water can reside is much closer to these faint stars. The result is a planet that may be exposed to violent space weather conditions for a good fraction of the age of the universe. 

Stellar space weather originates from the stellar corona. The transition from star to brown dwarf is characterized by increasing rotation speed and decreasing coronal emission. \textcolor{red}{At the same time, the observed surge} surge of radio emission indicates a transition from stellar corona to planet-like magnetosphere at the bottom of the main sequence~\citep{zarka1998auroral,pineda2017panchromatic}. Yet, brown dwarfs are regularly detected with energetic flares regardless of their expected low X-ray luminosity~\citep[e.g., ][]{hambaryan2004new, stelzer2006simultaneous, gizis2013kepler, paudel2020k2, schmidt2019largest}. How these flares are produced in the apparent absence of a solar-type corona is unclear~\citep{mullan2018frequencies}.%. An intermittent corona is conceivable, similarly XXXX. Perhaps, the rapid rotation of these low-mass objects plays a role, as it governs the dynamo process within, and consequently influences the complex process of flux emergence   

\citet{ilin2021giant} searched fully convective dwarfs that straddle the transition from star to brown dwarf with \textit{TESS}. They found four flares on four rapidly rotating ($P_{\rm rot}<10\,$h) M5-M7 dwarfs that lasted for multiple rotation periods of each star. The flares were observed to rotate in and out of view. The shape of this modulation together with the known inclination (combining $P_{\rm rot}$ and $v \sin i$ from high resolution spectroscopy) allowed us to determine the latitudes of these flares. These flares were found significantly closer to the pole than to the equator, in contrast to the Sun, where flares are usually found below $30^{\circ}$ latitude. Our results indicate a preference for flares at high latitudes in these stars, as a chance finding \textcolor{red}{of these latitudes among latitudinally equidistributed flares was } unlikely ($\sim 0.1\%$). If flares and associated particle eruptions indeed have a latitude preference, the space weather of a planet in an aligned orbit would be less severe than previously thought. %Are high flaring latitudes  an effect of the dynamo at high rotation speeds? Or is there something special about these stars' coronae that allowed them to produce flares at these latitudes? 

%Planets around late M dwarfs exists, see TRAPPIST-1 or LHS 3154~\citep{stefansson2023extreme}

\textcolor{red}{Among the studied objects, }TIC 277539431, TIC 277 for short, showed the highest latitude flare known to date at $\sim81^{\circ}$~\citep[][Table \ref{tab:starparams}]{ilin2021giant}. It is an M7 dwarf with an extremely short rotation period of $4.56\,$h. While the fast rotation and late spectral type could be indicative of \textcolor{red}{a diminishing corona, the detection of flares suggests otherwise.} %The star's age is unknown. \citet{schneider2018hazmat} classify it as a field star, but its FUV and NUV fluxes are on the high end for field stars and would, if not as high, be compatible with the NUV/FUV emission from younger open cluster members. 
%Is its high flaring latitude an effect of the dynamo at high rotation speeds? Or is there something special about this star's corona that led it to produce are flare at these latitudes? 

In this work, we follow TIC 277 with \textit{XMM-Newton} for 36 ks and use optical monitoring from \textit{TESS}~(Section~\ref{sec:obs}) to measure its coronal and flaring properties, respectively~(Section~\ref{sec:methods}). We contrast our results~(Section~\ref{sec:results}) with the literature to investigate if TIC~277 behaves like a low mass star or brown dwarf, investigate scenarios that are consistent with high latitude flare occurrence ~(Section~\ref{sec:discussion}), and \textcolor{red}{summarize our results} in Section~\ref{sec:summary}.


%producing a high-latitude flare requires an unusual corona, or if it is rather typical of stars of this spectral type and rotation.


\begin{table}
\footnotesize
\centering
    \caption{Stellar properties of TIC 277539431}
    \begin{tabular}{ll}\hline 
         Parameter & Value  \\\hline
         
         % $G$ mag & $14.7189 \pm 0.0011$ [4]\\
         % $G_{BP}$ mag & \\
         % $G_{BP}$ mag & \\
         % $K_s$ mag & $9.666 \pm 0.024$ [3]\\
         % $J$ mag & $10.630 \pm 0.023$ [3] \\
         Alternative ID & WISEA J105515.71-735611.3 \\
         Spectral type & M7 [1]\\
         Distance $d$ & $13.70\pm0.11\,$pc [2] \\
         Effective temperature $T_{\rm eff}$ & $2680^{60}_{-50}\,$ K [3]\\
         % age & \\
         Rotation period $P_{\rm rot}$ & $273.618 \pm 0.007\,$min [1]\\
         Projected rot. velocity $v\sin i$ & $38.6\pm1.0\,$km/s [1] \\
         Inclination $i$ & $(87.0^{+2.0}_{-2.4})^{\circ}$ [1]\\
         Radius $R$ & $0.145\pm0.004\,R_\odot$ [1]\\
         Bol. luminosity $L_{\rm bol}$ & \Lbol [*]\\\hline
        
    \end{tabular}
    \newline\footnotesize
    \flushleft
    [1] \citet{ilin2021giant}, 
    [2] \citet{bailer-jones2018estimating}, 
    % [3] 2MASS, \citet{skrutskie2006two}, 
    % [4] \citet{collaboration2016gaia,gaiacollaboration2023gaia}, 
    [3] \citet{pecaut2013intrinsic}
    [*] see Section~\ref{sec:methods:flareenergies}
    \label{tab:starparams}
\end{table}


\section{Observations}
\label{sec:obs}

\subsection{XMM-Newton}
\label{sec:obs:xmm}
\textit{XMM-Newton} is an X-ray telescope, which \textcolor{red}{was} launched into orbit in 1999, equipped with six science instruments that operate simultaneously, among them the European Photon Imaging Camera with three cameras, PN, MOS1 and MOS2; and the Optical Monitor (OM) that provides near-UV and optical photometry, among other capabilities.
We used \textit{XMM-Newton} to observe TIC 277 for 36 ks, i.e. two full rotation periods of the star, on August 5, 2022 (Proposal 090120; PI: E. Ilin). We used the EPIC instruments aboard \textit{XMM-Newton}, i.e., MOS1/2 and PN, that cover the soft X-ray band from $0.2$ and up to $12\,$keV, as well as the OM using its white light filter.

\subsubsection{Time series with EPIC and OM}

Fig.~\ref{fig:lightcurves} shows the time series using the combined PN and MOS time series together with the OM light curve. We extracted the events time series from EPIC using XMM SAS version 20\footnote{"Users Guide to the \textit{XMM-Newton} Science Analysis System", Issue 18.0, 2023 (ESA: \textit{XMM-Newton} SOC)} using the \texttt{evselect} task in the full $0.2-10\,$keV band. We selected a circular source region with 20 arcsec radius on all detectors, and a circular source-free background region 120 and 90 arcsec in MOS and PN, respectively. We used the \texttt{epiclccorr} task in XMM SAS to correct for energy and time dependent loss of events\footnote{\url{https://heasarc.gsfc.nasa.gov/docs/xmm/sas/help/epiclccorr/node3.html}, accessed on July 3, 2023}, and extracted a soft X-ray time series with a time binning of $200\,$s.

Simultaneously, we used the Optical Monitor (OM) aboard \textit{XMM-Newton}, to monitor TIC 277 in its white light filter at $10$\,s cadence. The OM white light filter is centered on $4060\,$\AA, with an equivalent width of $3470\,$\AA. \textcolor{red}{In each light curve, OM observed uninterruptedly for $4390\,$s, followed by a short gap of $318\,$s before the next one. We used the \texttt{omfchain} in XMM SAS to extract the 8 individual light curves. }



\begin{figure*}[ht!]
    \script{FIGURE_lightcurves.py}
    \begin{centering}
        \includegraphics[width=0.75\linewidth]{figures/lightcurves.png}
        \caption{
         Top panel: Optical Monitor light curve. Bottom panel: Background-subtracted X-ray light curve using the flux in the entire $0.2-10\,$keV range  (PN, MOS1 and MOS2 combined). The grey shaded portion defines the flare-only subset of the observations (see Section~\ref{sec:methods:epic} and Table~\ref{tab:specfit}).
        }
        \label{fig:lightcurves}
    \end{centering}
\end{figure*}


\subsubsection{EPIC Spectra}
From the EPIC observations, we used the standard spectral extraction procedure for point sources, as specified in the XMM\textcolor{red}{-Newton} SAS handbook. We extracted spectra using \texttt{evselect} to select events in the circular event and background regions, \texttt{epatplot} to check for pile-up of multiple photons in one exposure, \texttt{backscale} to calculate the area of the source region, and \texttt{rmfgen} and \texttt{arfgen} to convolve the observed energies with the instrument response, and produce the spectra. Finally, we use \texttt{grppha} to rebin the data in the spectra to obtain at least 15 counts per bin. To produce spectra for the quiescent and flaring portions of the observations separately, we used \texttt{tabgtigen} to select the time intervals to pass to \texttt{evselect}.

\subsection{TESS}
\label{sec:obs:tess}
Since 2018, \textit{TESS} has been supplying publicly available red-optical high-precision time series photometry in an ongoing all-sky survey. Each uninterrupted observing Sector provides an approximately 27 day long light curve in a broad $6000$-$10000\,$\AA band pass. 

We used \textcolor{red}{all} 5 Sectors of optical monitoring of TIC 277 in 2-minute cadence. Sector 12 was observed in May 2019, Sectors 37 and 39 in April and June 2021, and Sectors 64 and 65 in April and May 2023~(Fig.~\ref{fig:tess_lcs}). 



\section{Methods}
\label{sec:methods}
In this work, we analyse the coronal properties and flaring behavior of the late M dwarf TIC~277. From the X-ray observations with \textit{XMM-Newton}, we obtain a soft X-ray spectrum, which we model with two thermal emitter components~(Section~\ref{sec:methods:epic}). In TESS' optical time series photometry, we remove the variability introduced by the star's rotation, and find and characterize the flares in the so de-trended light curves~(Section~\ref{sec:methods:tess}). \textit{XMM-Newton} captured a flare simultaneously in both the PN and MOS instruments in X-ray, and marginally the OM instrument in the optical. We derive its soft X-ray energy using the luminosity derived from X-ray spectrum, and its bolometric energy from the OM data. For the following comparison between optical and X-ray flaring activity, we also calculate the bolometric energies of the flares detected in TESS~(Section~\ref{sec:methods:flareenergies}). 

\subsection{Spectral fitting: EPIC}
\label{sec:methods:epic}
The stellar corona can be described as an optically thin thermal plasma \textcolor{red}{that consists of multiple temperature components that represent different regions. The number of components one can identify in an X-ray spectrum depends on the brightness of the source.} We use XSPEC version 12.12.1~\citep{arnaud1996xspec} to fit a two-temperature (2T) additive \textcolor{red}{V}APEC~\citep{smith2001collisional,foster2012updated} model to the joint PN and MOS spectra, using solar abundances from~\citet{grevesse1998standard}, but adjusting to an Fe/0$=0.6$ ratio, which is more typical of M dwarfs~\citep{wood2018chandra}. We analyzed both the full data set, and the quiescent and flaring parts separately~\textcolor{red}{(see Fig.~\ref{fig:lightcurves})}. Neither subset could be adequately fit with a single temperature component (1T), and a 3T model did not improve the fit compared to the 2T-model. In the full data set, we used the \texttt{bayes} method in XSPEC to assign constant priors within the model's hard limits, and sample the uncertainties using the Markov-Chain-Monte-Carlo (MCMC) method using \texttt{chain} for a total of 30000 steps, discarding the first 5000 as the burn-in phase. From the same fit, but using PN data only, we derived the flux $F_X$ and X-ray luminosity $L_X$ in the soft $0.2-2\,$keV band. For the quiescent and flaring portions of the observations, we chose the same procedure, but used Gaussian prior distributions for the two coronal temperatures from the full data set, where we assigned the largest difference between the 50th, and 16th and 84th percentiles as standard deviation.

\subsection{Lightcurve de-trending and flare finding: TESS}
\label{sec:methods:tess}
We used AltaiPony~\citep{ilin2021altaipony} to de-trend \textcolor{red}{the Pre-search Data Conditioning Simple Aperture Photometry (PDCSAP)} light curves using the de-trending function \texttt{custom\_detrending}\footnote{see also the documentation online at \url{https://altaipony.readthedocs.io/en/latest/tutorials/detrend.html}, accessed on July 3, 2023}, detailed in~\citet{ilin2022searching}. The procedure begins with a 3rd order spline fitting of coarse trends, followed by iterative sine fitting to remove rotational modulation, and two Savitzky-Golay filters~\citep{savitzky1964smoothing} with a 6h and 3h window, respectively to remove aperiodic short term trends, while masking potential flare candidates in each step. The masking takes three steps each time. First, all outliers above $>2.5\sigma$ are masked. \textcolor{red}{Second, to mask the flare decay, we append the six-fold number of data points masked in the first step to the mask.} Third, three adjacent data points before and after the so expanded mask are added to the mask. 

In the de-trended light curves, we searched for flare candidates as at least 3 consecutive $>3\sigma$ outliers above the noise level\textcolor{red}{, defined as the rolling standard deviation of the de-trended light curve flux with a 2 h window}, filling in the noise level inside the previously masked areas with the mean of the noise levels adjacent to the mask. To the series of outliers, we kept \textcolor{red}{appending} data points to the flare candidate until a data point falls below $2\sigma$. 

\begin{figure*}
    \script{FIGURE_tess_lcs.py}
    \begin{centering}
        \includegraphics[width=\linewidth]{figures/tess_lcs.png}
        \caption{
         Normalized \textit{TESS} light curves. Black and red dots show the light curve with and without rotational variability and trends, respectively. The de-trended light curve is offset by $0.2$. The large, rotationally modulated, flare, localized at about $81^{\circ}$ latitude by~\citet{ilin2021giant}, appears in the second half of the top panel~(Sector 12).}
        \label{fig:tess_lcs}
    \end{centering}
\end{figure*}


\subsection{Flare energies: TESS, OM, and EPIC}
\label{sec:methods:flareenergies}
\textcolor{red}{The OM data contain one marginal flare that we interpret as associated with the large X-ray flare shortly after~(inset in Fig.~\ref{fig:lightcurves}). In the TESS light curves, we found a total of 18 flares. To assess the flaring activity of TIC~277, we computed the bolometric energies for both OM and TESS flares.} 

For the OM flare, we \textcolor{red}{used the} median flux of the lightcurve as the quiescent flux $F_0$, against which we measured the equivalent duration ($ED$) of the flare, i.e., the flare flux $F_{\rm flare}$, divided by $F_0$, integrated over the flare duration~\citep{gershberg1972results}:

\begin{equation}
\label{eq:ED}
ED=\displaystyle \int \mathrm dt\, \frac{F_{\rm flare}(t)}{F_0}.
\end{equation}

We then converted \textcolor{red}{the} $ED$ to flare energy using the bolometric flare energy following the procedure in~\citet{shibayama2013superflares}. We used the throughput curve for the white light filter scaled to unity at the peak of transmission as an optimistic response curve $R_{\lambda, \rm OM}$, given that the degradation of this filter's response is poorly constrained\footnote{\textit{XMM-Newton} User's Handbook 3.5.3.1, \url{https://xmm-tools.cosmos.esa.int/external/xmm_user_support/documentation/uhb/omfilters.html}, accessed Aug 8, 2023}. From that, we extracted a ratio $f_{\rm OM}$ of stellar to flare flux of $1.4\cdot 10^{-4}$ for a $10,000\,$K blackbody flare\footnote{A flare temperature of $T_{\rm flare}=10,000\,$K is a typical approximation for energetic M dwarf flares~\citep{kowalski2013timeresolved, howard2020evryflarea}.}:

\begin{equation}
    f_{\rm OM} = \frac{\displaystyle\int R_{\lambda, \rm OM} B_{\lambda(T_{\rm eff})}  d\lambda}{\displaystyle\int R_{\lambda, \rm OM} B_{\lambda(T_{\rm flare})} d\lambda} 
    \label{eq:ratio}
\end{equation}

With this, we can calculate the bolometric flare energy as:

\begin{equation}
    E_{\rm flare} = ED \cdot \pi R^2 \cdot \sigma_{kB} T_{\rm eff}^4 \cdot f_{\rm OM}
    \label{eq:eflare}
\end{equation}

We note that due to the optimistic response curve assumption, the resulting $E_{\rm flare}$ is likely higher in reality.

% To obtain the corrected $ED$, we then multiplied with $L_{\rm bol}$ to arrive at the bolometric flare energy. $L_{\rm bol}$ was derived using G and J magnitudes by converting J to V magnitude, then using the bolometric correction from \citep{mann2015how,mann2016erratum}.

Analogously to Eq.~\ref{eq:ratio} for the OM flare, we used the \textit{TESS} response curve to obtain a flux ratio $f_{\rm TESS}$ of $6.3\cdot10^{-3}$ in the optical filter of \textit{TESS}, and convert $ED$ to bolometric flare energy using Eq.~\ref{eq:eflare}. Besides the $ED$, \texttt{AltaiPony} also yields start and end time, defined as the difference between the first and last data point above the flare detection criterion defined in Section~\ref{sec:obs:tess}; and relative amplitude $a$, for each flare.

For the X-ray flare energy, we multiply flare duration $\Delta t$ by the flare luminosity in the EPIC data. The flare luminosity is equal to the difference between the quiescent and flaring X-ray luminosities in the $0.2-2.0\,$keV band, i.e.:

\begin{equation}
    E_{X, \mathrm{flare}} = \left(L_{X,\mathrm{flaring}} -  L_{X,\mathrm{quiescent}}\right) \cdot \Delta t
    \label{eq:xrayflare}
\end{equation}

\section{Results}
\label{sec:results}
With the \textit{XMM-Newton} and \textit{TESS} observations, we could constrain the coronal and flaring properties of TIC 277. We derive its coronal temperature and luminosity from the EPIC spectrum~(Section~\ref{sec:res:XrayTL}). We also measure the flare energies in X-rays from the EPIC instruments, and in optical from the OM, and compare them to the flare frequency distributions obtained from the 18 detected flares in \textit{TESS}~(Section~\ref{sec:res:flares}).


\subsection{Coronal temperature and luminosity}
\label{sec:res:XrayTL}
% From the proposal: $F_X=2.7^{-13}$ erg/s/cm$^2$, $L_X=6\cdot10^27$ erg/s (update with new eROSITA data) -- 1 oom above the measured XMM values.

From the \textit{XMM-Newton} EPIC spectra, we could measure the coronal properties of TIC~277 in both its quiescent, and flaring states~(Table~\ref{tab:specfit}). The resulting fit for the full data set~(Fig.~\ref{fig:spec_joint_all}) consists of a cooler and a hotter component of about \Tcool and \Thot\hspace{-1.cm}, respectively. Assuming these temperatures \textcolor{red}{for the prior distribution}, we also fitted 2T VAPEC models to the flaring and quiescent portions of the observations separately. The coronal temperatures did not update significantly using either subset. However, the emission measure weighted mean temperature $T_{\rm mean}$ changed from \Tqmean in the quiescent spectrum to \Tfmean in the flaring spectrum, since the hot component became more much dominant during the flare. From the quiescent PN spectra, we then also calculated soft X-ray flux of $F_X=$\FX, and a luminosity $L_X=$\LXquiet in the $0.2$-$2.0\,$keV band. %This suggests that the flaring corona emits the hotter component, while the cooler component corresponds to the quiescent corona. In this interpretation, the presence of a hot component in the quiescent spectrum indicates that TIC 277 is flaring below the detection threshold, consistent with an extrapolation of the flare frequency distribution to lower temperatures~(see the following Section, and Fig.~\ref{fig:ffd}). 

We note that not using the full set of observations as the prior resulted in an unconstrained hot coronal component during the flare. This can be interpreted as due to a lack of X-ray flux above $5\,$keV. %Hot flares with temperatures of $>50\,$MK have been measured on DS~Tuc~A~\citep{pillitteri2011xray}. On the Sun, the highest flare associated coronal temperatures are around $35\,$MK~\citep{kay2003soft}.


 \begin{table}
% \movetableright=-20mm
\footnotesize
\centering
    \script{TABLE_specfits.py}
    \caption{XSPEC fits to EPIC spectra for different subsets of observations. Fluxes and luminosities are given in the $0.2-2\,$keV band.}
    \input{output/mcmc_specfit.tex}
        \label{tab:specfit}
\end{table}


\begin{figure}
    \script{FIGURE_data_resid.py}
    \begin{centering}
        \includegraphics[width=\linewidth]{figures/joint_chain_fit_data_resid.png}
        \caption{
         \textcolor{red}{XMM-Newton EPIC spectra. The top panel shows the spectra taken with the different EPIC instruments together with the best-fit two-temperature VAPEC model with Fe/O$=0.6$ for the full data set.} The bottom panel shows the residuals to the fit.
        }
        \label{fig:spec_joint_all}
    \end{centering}
\end{figure}


\subsection{Flares}
\label{sec:res:flares}



\begin{table}
% \movetableright=-20mm
\centering
    \script{TABLE_tess_flares.py}
    \caption{Flares detected with \textit{TESS}. $t_s$ is the starting time of the flare, $a$ is its relative amplitude, and $E_{\rm bol}$ is the bolometric energy assuming a 10,000 K blackbody emission from the flare.}
    \input{output/tess_flares.tex}
        \label{tab:flares}
\end{table}


In \textit{TESS}, we found a total of 18 flares in the five Sectors, or equivalently, 125 days of observing at 2-min cadence. We fit the flare frequency distribution (FFD) with a power law of the form

\begin{equation}
    f(E) \mathrm{d} E = \beta \cdot E^{-\alpha} \mathrm{d} E.
\end{equation}
following the procedure in~\citep{ilin2021flares}, implemented as \texttt{FFD.fit\_powerlaw} using the \textit{mcmc} argument in \texttt{AltaiPony}~\citep{ilin2021altaipony}, which is using the posterior distribution derived in \citet{wheatland2004bayesian}, and sampling the posterior distribution with the \texttt{emcee} package~\citep{foreman-mackey2013emcee}.

The FFD power law fit converged after 13,500 steps on a slope $\alpha =$\ffdalpha\unskip. The frequency $R_{31.5}$ of flares per day above $\log_{10} E_{\rm flare} = 31.5\,\mathrm{erg}$ is about $\log_{10}R_{31.5}=$\ffdr\unskip. The slope is typical of other flaring stars, regardless of spectral type and rotation period~(see, e.g., Fig.~13 in \citealt{ilin2021flares}). $R_{31.5}$ is typical of flaring M dwarfs, in particular typical of late M dwarfs in the saturated activity regime, where $R_{31.5}$ becomes independent of rotation period~\citep{medina2020flare,murray2022study}. 

For the X-ray flare in Fig.~\ref{fig:lightcurves}, we obtain a total energy of \eepic in the $0.2-2\,$keV band, using $L_{X,\rm quiescent}=$\LXquiet\unskip, $L_{X,\rm flaring}=$\LXflaring\unskip, and $\Delta t =$\xraydur\unskip (see Fig.~\ref{fig:lightcurves}) in Eq.~\ref{eq:xrayflare}. The energy of the corresponding OM flare is \eom\unskip. We use the 36ks observing baseline of \textit{XMM-Newton} to calculate a flare rate in OM. Fig.~\ref{fig:ffd} illustrates that it is roughly consistent, if slightly higher, than the extension of \textit{TESS}' flare frequency distributions to lower energies. About one flare per day of the order of the observed one or above can therefore be expected in future X-ray observations of TIC~277. However, this assumes that all optical flares above this energy in the optical data correspond to detected X-ray flares in the EPIC instruments, which is not always the case~\citep{paudel2021simultaneous}. \citet{guarcello2019simultaneous, kuznetsov2021stellar} find moderate correlation between flare energies in Kepler/K2 data and \textit{XMM-Newton}, but the majority of their sampled flare energies are two orders of magnitude above the flare in this work, both in X-ray and optical. \citet{kuznetsov2021stellar} compare primary Kepler mission observations with \textit{XMM-Newton} to find a similar result. We note that the X-ray flare light curve follows the typical fast-rise exponential decay shape, and is therefore unlikely to be eclipsed by the star~\citep{johnstone2012soft}.


\begin{figure}
    \script{FIGURE_tess_ffd.py}
    \begin{centering}
        \includegraphics[width=\linewidth]{figures/277539431_tess_ffd.png}
        \caption{
         Cumulative flare frequency distribution of \textit{TESS} flares (black dots), with a power law fit (green line), and uncertainties (dotted lines). The rate of OM flares (gray square) is calculated from the observing baseline of \textit{XMM-Newton}. 
        }
        \label{fig:ffd}
    \end{centering}
\end{figure}

\section{Discussion}
\label{sec:discussion}
% \citep{pass2022constraints} on rotational evol of late Ms

TIC~277's spectral type and rotation places it in the middle of the transition from star to brown dwarf. Its coronal properties are therefore not known a priori. Our findings indicate that TIC 277's coronal temperature~(Section~\ref{sec:discussion:xraytemp}) and flaring activity~(Section~\ref{sec:discussion:flares}) are typical of saturated fully convective M dwarf stars. However, TIC 277 is rotating rapidly, even for the typically fast rotating late M dwarfs~\citep{medina2022galactic}, which is more typical of brown dwarfs. This combination could lead to an unusual manifestation of the stellar dynamo that produces high latitude flares that other dynamo states cannot generate~\citep{weber2016modeling, weber2023understanding}. TIC 277 is \textcolor{red}{fainter in X-ray by about an order of magnitude than M dwarfs on the canonical saturated branch}, which, in combination with the detection of a high latitude flare, may be an indication of polar updraft migration that is suspected to occur in rapid rotators~(Section~\ref{sec:discussion:xraylum}).


\subsection{Coronal temperature}
\label{sec:discussion:xraytemp}
Low mass stars' coronae are not uniform. Magnetic structures create a range of temperatures, starting at about 1 MK by definition and rising up to 30 MK in flares, in the heterogeneous Solar corona~\citep{vaiana1978recent}. Stellar point source X-ray observations are a superposition of various temperatures and densities. The resolution of different temperature components depends on the apparent brightness and integration time of the spectrum. Nearby stars like Proxima Cen can be modeled with a range of temperatures~\citep[e.g.,][]{gudel2004flares, drake2020pointing}, while one can often only capture few dominant temperatures in fainter objects.  

We find that the quiescent corona of TIC 277 is best described by two components, similar to many other active M dwarfs, in line with the standard picture for stellar flares~\citep{wargelin2008xray, robrade2010quiescent, behr2023muscles, magaudda2022firsta}. Both the cooler and the hotter component represent thermal radiation from hot plasma that evaporates from the chromosphere into the corona. The hotter component is driven by particles precipitating downward during flares that then heat the chromosphere to high temperatures, and cause additional evaporation into the coronal flaring loops~\citep{benz2016flare}. In our osbervation, this interpretation is supported by the observed delay between the optical and the X-ray flare~\citep[Fig.~\ref{fig:lightcurves}, ][]{hawley2003multiwavelength}. The optical flare originates from the lower chromosphere, which is heated by the precipitating particles. The heating then causes evaporation of particles into the flare loop. In the cooler component, the source of the more quiescent coronal heating is an open question, even for the Sun, where the respective contributions of reconnection and magnetohydrodynamical waves are under investigation~\citep{vandoorsselaere2020coronal}.  %A third, very hot, possibly $>40\,$MK component could only be poorly constrained due to a lack of sensitivity to flux above $5\,$keV, and its short lifetime during the peak of the flare.

The range of 6.5-11 MK for the emission measure weighted temperature of TIC 277 places it squarely in the saturated regime with other mid M dwarfs down to spectral type M6~\citep{wright2018stellar, magaudda2020relation, stelzer2022first,robrade2005xray,raassen2003xray,paudel2021simultaneous, foster2020corona}. Flares on the slowly rotating Proxima Cen (M5.5, $P_{\rm rot} = 83\,$d,~\citealt{anglada-escude2016terrestrial}), appear with a range of temperatures at 2-3 MK and 10-20 MK~\citep{gudel2004flares, fuhrmeister2011multiwavelength, fuhrmeister2022high, howard2022mouse} placing it in the transition region between saturated and unsaturated regimes. Only at rotation periods beyond 100 days, flaring activity declines to yield coronal temperatures $<2\,$MK in M3-M6 dwarfs~\citep{wright2018stellar, foster2020corona}. 

There are only few late M dwarfs beyond spectral type M6 with a spectrally resolved corona. Their intrinsic faintness and gradual decline in coronal emission toward the brown dwarf regime make them hard to detect~\citep{berger2010simultaneous, cook2014trends, stelzer2022first}. One example is NLTT 33370 AB, an M7 binary with a rotation period of about $3.8\,$h (close to TIC 277's $4.56\,$h, and of the same spectral type) is combined of a 3.1 MK and 14 MK component~\citep{williams2015simultaneous}. Another is TRAPPIST-1~(M8, $P_{\rm rot} = 3.3\,$d~\citep{luger2017sevenplanet}). Its corona appears cooler, with a 1.74 MK and a 9.6 MK component~\citep{wheatley2017strong}, and an emission measure weighted temperature of 5.36 MK~\citep{brown2023coronal}. 

Bearing in mind the relatively low two-temperature resolution of TIC~277's X-ray spectrum and the dearth of X-ray spectra for stars at the bottom of the main sequence, it shows a coronal temperature make-up common for mid-to-late M dwarf in the saturated activity regime that it will possibly keep for gigayears until it spins down to very low rotation rates~\citep{medina2022galactic, engle2023living}. 

%DS Tuc A, a young G dwarf, has been observed flaring with a >50 MK temperature in the flare peaks~\citep{pillitteri2022xraya}, and PMS stars can exceed 100 MK~\citep{getman2008xray, getman2021xray}. The Sun can heat the corona up to over 30 MK during flares~\citep{benz2016flare}. A flare on AD Leo (M3.5) had a peak temperature of about 37 MK~\citep{stelzer2022great}. CC Eri, a K7 dwarf, was described with a 3T model~\citep{crespo-chacon2007xray} with energies about $10^{33}\,$erg and peak temperatures of 30 MK.
\subsection{Flaring activity}
\label{sec:discussion:flares}

TIC~277's flare rate is consistent with other saturated fully convective dwarfs~\citep{medina2020flare, murray2022study}, and so is the energy distribution with a power law slope of \ffdalpha\unskip. It is in fact comparable to TRAPPIST-1's flaring behavior~\citep{paudel2018k2}, which has been investigated for its effects on the seven terrestrial planets in its orbit. 

%Recent observations of TRAPPIST-1 b~\citep{greene2023thermal, ih2023constraining} with the James Webb Space Telescope find the planet mostly stripped of an atmosphere, which the authors attribute to the intense flare driven X-ray radiation and energetic particle exposure at the short orbital periods of the innermost planet. For the habitable zone planets, TRAPPIST-1 e and f, simulations suggest that flares alter their chemical composition, in particular ozone, nitric oxide and hydroxide, but that the effects would be below the noise floor of JWST~\citep{chen2021persistence}. However, given the uncertain age of TRAPPIST-1~\citep{burgasser2017age, gonzales2019reanalysis, birky2021improved}, the temperate planets e and f may have been entirely stripped of their atmospheres in the past gigayear(s)~\citep{garraffo2017threatening}. 

TIC~277's rapid rotation and UV flux is indicative of a younger age than TRAPPIST-1, although it has not been attributed to any star cluster so far~\citep{schneider2018hazmat}. Interpreting the upcoming observations of TRAPPIST-1 e, one of the rocky planets in the habitable zone of the system, with the James Webb Space Telescope will most likely involve its activity history. Studies of its younger counterparts, late M dwarfs like TIC~277, are required to empirically constrain cumulative effects of atmospheric forcing of planets. If its energetic flares commonly occur at high latitudes~\citep{ilin2021giant}, future models of energetic particle exposure of habitable zone planets may have to include age dependent latitudes of particle eruption to reproduce observations. 

\subsection{X-ray luminosity}
\label{sec:discussion:xraylum}

% \begin{figure}
%     \script{FIGURE_LXLBol_rot.py}
%     \begin{centering}
%         \includegraphics[width=\linewidth]{figures/lx_lbol.png}
%         \caption{$L_X/L_{\rm bol}$ over rotation period of stars of different spectral types. For partially convective dwarfs, and fully convective mid-M dwarfs, we show the results from~\citet{wright2011stellaractivityrotation, wright2016solartype}, color-coded by $V-K$ magnitude. Late M dwarfs from the literature are marked with diamonds, and are detailed in Table~\ref{tab:latmlxlbol}. TIC~277 is shown as red star. The few late M dwarfs with known rotation period and X-ray luminosity show a tentative decline in $L_X/L_{\rm bol}$ relative to earlier fully convective dwarfs, which might be due to polar updraft migration~(see Section \ref{sec:discussion:xraylum}).
%         }
%         \label{fig:lx_lbol}
%     \end{centering}
% \end{figure}

TIC~277's coronal luminosity relative to bolometric luminosity of $L_X/L_{\rm bol}=$\LXLbol in quiescence is about an order of magnitude lower than \textcolor{red}{the canonical} $L_X/L_{\rm bol}\sim 10^{-3}$ average in the saturated regime of partly and fully convective M dwarfs\citep{wright2011stellaractivityrotation,wright2016solartype,wright2018stellar}.% In the Hyades and Praesepe, \citet{nunez2022factory} cover both fully convective and rapidly rotating M dwarfs, and find a downward trend in $L_X/L_{\rm bol}$ with decreasing rotation period for M3.5-M7 dwarfs. This tentatively points toward a supersaturation of the corona at fast rotation speeds, as has been suggested before for M~dwarfs~\citep{jeffries2011investigating}, but also earlier type stars~\citep{prosser1996rosat, argiroffi2016supersaturation} and even red giants~\citep{dixon2020rotationally}. However, none of the aforementioned M~dwarf studies probe below spectral type M6. 

At later spectral type, a decline in coronal activity is expected as the magnetosphere transitions from a stellar to a planetary one~\citep{pineda2017panchromatic}. \citet{stelzer2022first} investigated late M and L dwarfs with eROSITA to show that $L_X/L_{\rm bol}$ increases towards later spectral types~\citep{magaudda2022firsta}, but begins to decrease again from M7 downward. In their compilation, together with previous work from \citet{stelzer2012ultracool, cook2014trends, deluca2020extras, williams2014trends, berger2008simultaneous}, TIC~277 appears at the lower end\textcolor{red}{, but consistent with} the $L_X/L_{\rm bol}$ distribution for M7 dwarfs. However, the rotation periods are unknown for most of these late M dwarfs. Sometimes projected rotation periods are known~\citep{cook2014trends}, but true periods are only detected in a handful of these stars, which we list in Table~\ref{tab:latmlxlbol}. TIC~277 appears \textcolor{red}{consistent with} this small sample.

\begin{table*}[]
    \centering
    \caption{Late M dwarfs with known rotation period and X-ray luminosity.}
    \begin{tabular}{lcccc}
        ID & SpT &$P_{\rm rot}$ [d] & $\log_{10} \frac{L_X}{L_{\rm bol}} $ & energy range [keV]\\\hline
        TIC 277 (this work) & M7 & 0.19 & $-4$ & $0.2-2$ \\
        TRAPPIST-1 [1] & M8 & 3.3 & $-4$ & $0.3-10$\\
        LHS 248 [2] & M6.5 & 0.46 & $-3.9$ & $0.2-2$\\
        NLTT 33370 AB [3] & M7 & 0.16 & $-3.5$ & $0.2-2$ \\
        LP 412-31  [4] & M8 & 0.61 & $-3.1$& \textcolor{red}{soft X-ray band?}\\\hline
    \end{tabular}
     \footnotesize
     \vspace{0.1cm}\newline
    [1] \citet{brown2023coronal}, 
    [2] \citet{cook2014trends}, 
    [3] \citet{williams2015simultaneous}, 
    [4] \citet{stelzer2006simultaneous}    
    %3.3 d from \citep{luger2017sevenplanet}
    \label{tab:latmlxlbol}
\end{table*}


%At the same time, strong magnetic fields are found across the board in brown dwarfs~(XXXX).

\textcolor{red}{In late M dwarfs like TIC~277, the decline in X-ray activity is expected as a result of decreasing efficiency in coronal heating in the transition region between a stellar and a brown dwarf atmosphere~(e.g.,~\citealt{williams2014trends}). However, the flaring activity and coronal temperature of TIC~277 are in line with saturated activity, implying that the efficiency of flare production has not diminished.} An alternative explanation, that would at the same time explain the occurrence of a high latitude flare in \textit{TESS}, is polar updraft migration~\citep{stepien2001rosat}. Polar updraft migration implies that at high rotation rates, flux emergence becomes more efficient near the poles than at the equator~\citep{weber2016modeling}\textcolor{red}{, producing flares there, but} draining the equatorial regions of magnetic flux required to produce a corona \textcolor{red}{at low latitudes}. As a result, the X-ray luminosity diminishes. Since TIC~277 is seen nearly equator-on~(Table~\ref{tab:starparams}), polar updraft could explain its fainter corona. Considering the high-latitude flare, viewing geometry, star-like flaring behavior and \textcolor{red}{star-like} coronal temperatures together, we \textcolor{red}{suggest this scenario as an alternative to the brown dwarf transition explanation}, that would invoke a diminishing corona due to the overall cooler, more neutral atmosphere. 



%As often, late M dwarfs are underrepresented in these studies. 

% While it is not clear whether M7 falls in the stellar regime of~\citet{reiners2022magnetism} or into that latter brown dwarf realm, or has features of both, it appears at the lower bound regardless. The small number of supersaturated stars in \citet{nunez2022factory}  can only tentatively suggest that this is a common phenomenon among mid-to-late M dwarfs when they are quickly rotating.  

%Compared to the X-ray luminosities of M7 dwarfs among the late M dwarfs observed by eROSITA in \citet{stelzer2022first}, TIC 277 appears mildly . This could be an effect of its fast rotation, and consequent supersaturation as in \citet{magaudda2022first}, but we lack the rotation periods to confirm this.



 

% \begin{figure}
%     \script{FIGURE_r315_comparison.py}
%     \begin{centering}
%         \includegraphics[width=\linewidth]{figures/r315_prot.png}
%         \caption{
%          R31.5 against rotation period for fully convective M dwarfs.
%         }
%         \label{fig:r315}
%     \end{centering}
% \end{figure}







\section{Summary and Conclusions}
\label{sec:summary}
We investigated whether TIC 277, a rapidly rotating M7 dwarf with a flare localized at $81^{\circ}$ latitude, shows unusual coronal or flaring properties that could explain the occurrence of this flare so close to the rotational pole. We obtained 36 ks of EPIC and OM observations with \textit{XMM-Newton}, and studied the five Sectors of red-optical 27-day light curves provided by \textit{TESS}. We found a mean quiescent coronal temperature of \Tqmean, and a flaring rate $f(> \log_{10} E_{\rm flare, bol} = 31.5\,\rm{erg})\approx$\ffdr with energy distribution with a slope of \ffdalpha, all typical of fully convective M dwarfs in the saturated regime. We also detected a simultaneous X-ray and optical flare with an energy of \eepic in the $0.2-2\,$keV band, and bolometric energy of \eom from the OM observations. In combination with the 18 detected \textit{TESS} flares, we estimate that X-ray flares on TIC~277 can be observed with \textit{XMM-Newton} at least about once a day. %The bolometric and soft-X-ray flare energies are within an order of magnitude within each other, which places them among other stars with simultaneous X-ray and optical flare detections, although, typically, the white light emission dominates over soft X-ray~\citep{guarcello2019simultaneous, kuznetsov2021stellar}.

An indication of an unusual corona stems from its X-ray luminosity relative to bolometric luminosity $L_X/L_{\rm bol}=$\LXLbol, which is about an order of magnitude lower than is typical for saturated activity M dwarfs. TIC 277 is an extreme rapid rotator with a rotation period of only $4.56\,$h. The detected high-latitude flare in \textit{TESS}~\citep{ilin2021giant} may hence be a product of high-latitude flux emergence driven by its fast rotation~\citep{weber2016modeling,weber2017suppression}. This effect is closely related to polar updraft migration, which suppresses coronal emission in equatorial regions~\citep{stepien2001rosat}. However, less than ten flare latitudes on stars other than the Sun are known to date~\citep{wolter2008doppler, ilin2021giant, johnson2021simultaneous}. Localization of more flares on the surfaces of a broader range of stars will allow us to resolve if polar updraft indeed occurs in these stars.

%If TIC~277 is relatively young, as its NUV/FUV fluxes at the high end for field stars suggest~\citep{schneider2018hazmat}, the space weather it creates may be representative of that in the youth of similar very low mass planet hosts. These hosts, such as M8 type TRAPPIST-1~\citep{gillon2017seven}, or M6 type LP 890-9~\citep{delrez2022two}, are the most favorable targets for transmission spectroscopy of terrestrial planet atmospheres for the JWST. As the makeup of their planets' atmospheres strongly depends on the high energy radiation and particle environment created by their host star~\citep{airapetian2020impact}, their activity history as traced by TIC~277 is crucial for the interpretation of present~(XXXX) and future observations.


\section*{Acknowledgements}
 E.I. acknowledges funding from the Deutsche Luft- und Raumfahrtgesellschaft (FKZ 50 OR 2209, project \textit{X-ray Loops}). K.P. acknowledges funding from the German \textit{Leibniz-Gemeinschaft} under project number P67/2018. This work is based on observations obtained with \textit{XMM-Newton}, an ESA science mission with instruments and contributions directly funded by ESA Member States and NASA. This work made use of the open source python software packages \texttt{lightkurve}\citep{lightkurvecollaboration2018lightkurve}, \texttt{astropy}~\citep{robitaille2013astropy}, \texttt{numpy}~\citep{harris2020array}, \texttt{pandas}~\citep{reback2022pandasdev}, \texttt{matplotlib}~\citep{hunter2007matplotlib}, \texttt{emcee}~\citep{foreman-mackey2013emcee}, \texttt{scipy}~\citep{mckinney2010data}, and \texttt{altaipony}~\citep{ilin2021altaipony}. This work has made use of data from the European Space Agency (ESA) mission
{\it Gaia} (\url{https://www.cosmos.esa.int/gaia}), processed by the {\it Gaia}
Data Processing and Analysis Consortium (DPAC,
\url{https://www.cosmos.esa.int/web/gaia/dpac/consortium}). Funding for the DPAC
has been provided by national institutions, in particular the institutions
participating in the {\it Gaia} Multilateral Agreement.

\section*{Data Availability}
All data used in this study are publicly available in their respective archives.
This study used the reproducibility software \href{https://github.com/showyourwork/showyourwork}{showyourwork}
\citep{luger2021mappinga} to create all figures, format all tables, and compile the manuscript. Each figure links to the dataset stored on Zenodo (link to be added after acceptance), and to the script that produced the figure in the git repository on GitHub:~\href{https://github.com/ekaterinailin/xmm_for_tic277}{github.com/ekaterinailin/xmm\_for\_tic277}. The data analysis scripts can be found under \href{https://github.com/ekaterinailin/tic277}{github.com/ekaterinailin/tic277}. 

\bibliography{references}

\end{document}
